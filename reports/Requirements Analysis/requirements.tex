\documentclass{article}
\usepackage{graphicx}
\usepackage{wrapfig}

\begin{document}

\title{TI2800 Contextproject - Cultural Heritage \\ Requirements Analysis}
\author{Sjoerd van Bekhoven \\ Tim Eversdijk \\ Herman Banken \\ Rutger Plak \and 1234567 \\ 1234567 \\ 1234567 \\ 1358375}
\maketitle

\section{Stakeholder Identification}
Het systeem kan door meerdere soorten gebruikers worden benut:
\begin{itemize}
	\item{Toeristen}
	\begin{itemize}
		\item{Dagje uit}
		\item{Tour plannen}
		\item{Weekendje weg}
		\item{etc.}
	\end{itemize}
	\item{Onderzoekers}
	\begin{itemize}
		\item{Archeologen}
		\item{(Kunst)-historici}
		\item{etc.}
	\end{itemize}
	\item{Studenten}
	\begin{itemize}
		\item{Meer leren}
		\item{Werkstuk maken}
		\item{Verdiepen}
		\item{etc.}
	\end{itemize}
\end{itemize}
Iedere gebruiker zal andere behoeften hebben die grotendeels overlappen. Bladiebla, dit dat zus zo. Hier moet een verhaal over wie nu toeristen zijn, wat ze doen, bla die bla.
\section{System requirements}
De vereisten van het te maken systeem verdelen we onder in verschillende kopjes.
\begin{enumerate}
	
	\item{Customer Requirements}
	Statements of fact and assumptions that define the expectations of the system in terms of mission objectives, environment, constraints, and measures of effectiveness and suitability (MOE/MOS). The customers are those that perform the eight primary functions of systems engineering, with special emphasis on the operator as the key customer. Operational requirements will define the basic need and, at a minimum, answer the questions posed in the following listing:[1]
	\begin{enumerate}
		\item{Operational distribution or deployment:} Where will the system be used?
		\item{Mission profile or scenario:} How will the system accomplish its mission objective?
		\item{Performance and related parameters:} What are the critical system parameters to accomplish the mission?
		\item{Utilization environments:} How are the various system components to be used?
		\item{Effectiveness requirements:} How effective or efficient must the system be in performing its mission?
		\item{Operational life cycle:} How long will the system be in use by the user?
		\item{Environment:} What environments will the system be expected to operate in an effective manner?
	\end{enumerate}
	\item{Architectural Requirements}
	Architectural requirements explain what has to be done by identifying the necessary system architecture of a system.
	\item{Structural Requirements}
	Structural requirements explain what has to be done by identifying the necessary structure of a system.
	\item{Behavioral Requirements}
	Behavioral requirements explain what has to be done by identifying the necessary behavior of a system.
	\item{Functional Requirements}
	Functional requirements explain what has to be done by identifying the necessary task, action or activity that must be accomplished. Functional requirements analysis will be used as the toplevel functions for functional analysis.[1]
	\item{Non-functional Requirements}
	Non-functional requirements are requirements that specify criteria that can be used to judge the operation of a system, rather than specific behaviors.
	\item{Performance Requirements}
	The extent to which a mission or function must be executed; generally measured in terms of quantity, quality, coverage, timeliness or readiness. During requirements analysis, performance (how well does it have to be done) requirements will be interactively developed across all identified functions based on system life cycle factors; and characterized in terms of the degree of certainty in their estimate, the degree of criticality to system success, and their relationship to other requirements.[1]
	\item{Design Requirements}
	The “build to,” “code to,” and “buy to” requirements for products and “how to execute” requirements for processes expressed in technical data packages and technical manuals.[1]
	\item{Derived Requirements}
	Requirements that are implied or transformed from higher-level requirement. For example, a requirement for long range or high speed may result in a design requirement for low weight.[1]
\item{Allocated Requirements}
	A requirement that is established by dividing or otherwise allocating a high-level requirement into multiple lower-level requirements. Example: A 100-pound item that consists of two subsystems might result in weight requirements of 70 pounds and 30 pounds for the two lower-level items.[1]
	
\end{enumerate}
\subsection{Strengths}
\begin{itemize}
\item blabla?
\end{itemize}
\subsection{Weaknesses}
\begin{itemize}
\item Cold start: in het begin is er nog nauwelijks user data beschikbaar behalve die van Flickr. De recommendations zijn daarom nog nauwelijks gebaseerd op user-input. blabla.
\item Long tail: veel ongebruikte data. Nooit bekeken monumenten worden ook later niet meer gevonden. blabla.
\end{itemize}
\section{Measurable goals}
Lorem ipsum dolor sit amet, consectetur adipisicing elit, sed do eiusmod tempor incididunt ut labore et dolore magna aliqua. Ut enim ad minim veniam, quis nostrud exercitation ullamco laboris nisi ut aliquip ex ea commodo consequat. Duis aute irure dolor in reprehenderit in voluptate velit esse cillum dolore eu fugiat nulla pariatur. Excepteur sint occaecat cupidatat non proident, sunt in culpa qui officia deserunt mollit anim id est laborum.
\section{Prototypes}
Lorem ipsum dolor sit amet, consectetur adipisicing elit, sed do eiusmod tempor incididunt ut labore et dolore magna aliqua. Ut enim ad minim veniam, quis nostrud exercitation ullamco laboris nisi ut aliquip ex ea commodo consequat. Duis aute irure dolor in reprehenderit in voluptate velit esse cillum dolore eu fugiat nulla pariatur. Excepteur sint occaecat cupidatat non proident, sunt in culpa qui officia deserunt mollit anim id est laborum.
\section{Use cases}
Lorem ipsum dolor sit amet, consectetur adipisicing elit, sed do eiusmod tempor incididunt ut labore et dolore magna aliqua. Ut enim ad minim veniam, quis nostrud exercitation ullamco laboris nisi ut aliquip ex ea commodo consequat. Duis aute irure dolor in reprehenderit in voluptate velit esse cillum dolore eu fugiat nulla pariatur. Excepteur sint occaecat cupidatat non proident, sunt in culpa qui officia deserunt mollit anim id est laborum.
\section{Software requirements specification}

\end{document}