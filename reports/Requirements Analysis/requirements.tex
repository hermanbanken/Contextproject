\documentclass{article}
\usepackage{graphicx}
\usepackage{wrapfig}
\begin{document}

\title{TI2800 Contextproject - Cultural Heritage \\ Requirements Analysis}
\author{Sjoerd van Bekhoven \\ Tim Eversdijk \\ Herman Banken \\ Rutger Plak \and 1234567 \\ 1234567 \\ 1234567 \\ 1358375} 
\maketitle

\section{Stakeholder Identification}
Het systeem zal voorzien in informatie over en verbanden tussen monumenten, maar zal ook informatie geven over wat er om de monumenten heen vindbaar is. Hierbij kan gedacht worden aan caf\'es, restaurants of openbare voorzieningen. Daarnaast is er de mogelijkheid om een route uit te stippelen. De beoogde doelgroep bestaat dus toeristen die bijvoorbeeld:
\begin{itemize}
		\item{een intellectueel dagje uit in een bepaalde stad willen plannen.}
		\item{een weekend door Nederland willen toeren langs verschillende monumenten.}
		\item{informatie over een bepaald monument willen opzoeken.}
		\item{willen opzoeken in welke stad de interessantste monumenten te bezoeken zijn.}
		\item{een monument gaan bezoeken en willen weten waar ze daarna wat kunnen eten of drinken.}
		\item{een monument hebben bezocht en graag inspiratie willen opdoen over andere gerelateerde of soortgelijke monumenten.}
\end{itemize}
Volgens de World Tourism Organization is een toerist iemand "die reist naar plaatsen buiten zijn/haar gebruikelijk milieu, die niet meer dan \'e\'en jaar voor vrije tijd, zaken en andere doeleinden blijft en die niet beloond wordt voor zijn/haar activiteit ter plaatse".\footnote{http://pub.unwto.org/WebRoot/Store/Shops/Infoshop/Products/1034/1034-1.pdf, World Tourism Organization. 1995. p. 14. Retrieved 2009-03-26} \\
Het systeem zal voornamelijk worden toegespitst op toeristen, maar zal ook voor kenners, wetenschappers, scholen, studenten e.d. een interessante informatiebron worden. Naast toeristen, zullen ook zij het voor hen relevante deel van het systeem kunnen gebruiken.
\section{System requirements}
De vereisten van het te maken systeem verdelen we onder in verschillende kopjes.
\begin{enumerate}
	
	\item{\textbf{Customer Requirements}}
	Statements of fact and assumptions that define the expectations of the system in terms of mission objectives, environment, constraints, and measures of effectiveness and suitability (MOE/MOS). The customers are those that perform the eight primary functions of systems engineering, with special emphasis on the operator as the key customer. Operational requirements will define the basic need and, at a minimum, answer the questions posed in the following listing:[1]
	\begin{enumerate}
		\item{\textbf{Operational distribution or deployment:}} Where will the system be used?
		
		Het systeem zal gebruikt worden in de toeristische sector. Toeristen kunnen via de webinterface zoeken naar een monument om te bezoeken. Het systeem zal dus in de huiselijke sfeer worden gebruikt, maar ook mobiel, wanneer de toerist bij het monument aankomt en de achtergrond informatie nog bij de hand wil hebben. Ook onderweg zal hij het systeem willen gebruiken.
		\item{\textbf{Mission profile or scenario:}} How will the system accomplish its mission objective? 
		
		Het systeem is opgedeeld in een webinterface en een back-end. De back-end verzamelt en berekent alle data die de webinterface kan gebruiken. Deze back-end gebruikt een dataset met alle monumenten in het Rijksmonumenten register en bij behorende afbeeldingen uit Wikimedia Commons. Daarnaast verzameld deze extra input gegenereerd door gebruikers zoals reacties op afbeeldingen en pagina\'s. Op deze manier weet de back-end te bepalen of monumenten relevant zijn of niet. 
		
		\item{\textbf{Performance and related parameters:}} What are the critical system parameters to accomplish the mission?
		
		De beschikbare data is een kritische systeem parameter omdat het van de data afhangt wat het systeem de gebruiker kan laten zien. Wanneer het systeem de gebruiker geen relevante informatie kan geven dan is de missie gefaald.
		
		\item{\textbf{Utilization environments:}} How are the various system components to be used?
		
		De webinterface bestaat uit diverse onderdelen. Allereerst is er een kaart met daarop de gefilterde monumenten. Deze kaart is te gebruiken om snel een geografische selectie te maken van de monumenten. Er is een filter-component waar gebruikers aan de hand van gegeven labels danwel berekende selectiecriteria monumenten kunnen filteren. 
		
		Het detailoverzicht van een monument bevat informatie over het monument zelf. De toerist kan zich zo vast verdiepen in het monument. Daarnaast bevat deze pagina aggregated data van weersinformatie, hotelboeking-sites, etc.
		
		\item{\textbf{Effectiveness requirements:}} How effective or efficient must the system be in performing its mission?
		\item{\textbf{Operational life cycle:}} How long will the system be in use by the user?
		\item{\textbf{Environment:}} What environments will the system be expected to operate in an effective manner?
	\end{enumerate}
	\item{\textbf{Architectural Requirements}}
	Architectural requirements explain what has to be done by identifying the necessary system architecture of a system.
	\item{\textbf{Behavioral Requirements}}
	Behavioral requirements explain what has to be done by identifying the necessary behavior of a system.
	\item{\textbf{Performance Requirements}}
	The extent to which a mission or function must be executed; generally measured in terms of quantity, quality, coverage, timeliness or readiness. During requirements analysis, performance (how well does it have to be done) requirements will be interactively developed across all identified functions based on system life cycle factors; and characterized in terms of the degree of certainty in their estimate, the degree of criticality to system success, and their relationship to other requirements.[1]	
\end{enumerate}
\subsection{Strengths}
\begin{itemize}
\item blabla?
\end{itemize}
\subsection{Weaknesses}
\begin{itemize}
\item Cold start: in het begin is er nog nauwelijks user data beschikbaar behalve die van Flickr. De recommendations zijn daarom nog nauwelijks gebaseerd op user-input. blabla.
\item Long tail: veel ongebruikte data. Nooit bekeken monumenten worden ook later niet meer gevonden. blabla.
\end{itemize}
\section{Measurable goals}
\begin{itemize}
\item Het is een goal om zo hoog mogelijke scores te halen met het vinden van vergelijkbare monumenten op basis van hun visuele overeenkomsten in de afbeeldingen.
\item Dit moet meer worden, bladiebla.
\end{itemize}
\section{Prototypes}
Lorem ipsum dolor sit amet, consectetur adipisicing elit, sed do eiusmod tempor incididunt ut labore et dolore magna aliqua. Ut enim ad minim veniam, quis nostrud exercitation ullamco laboris nisi ut aliquip ex ea commodo consequat. Duis aute irure dolor in reprehenderit in voluptate velit esse cillum dolore eu fugiat nulla pariatur. Excepteur sint occaecat cupidatat non proident, sunt in culpa qui officia deserunt mollit anim id est laborum.
\section{Use cases}
Lorem ipsum dolor sit amet, consectetur adipisicing elit, sed do eiusmod tempor incididunt ut labore et dolore magna aliqua. Ut enim ad minim veniam, quis nostrud exercitation ullamco laboris nisi ut aliquip ex ea commodo consequat. Duis aute irure dolor in reprehenderit in voluptate velit esse cillum dolore eu fugiat nulla pariatur. Excepteur sint occaecat cupidatat non proident, sunt in culpa qui officia deserunt mollit anim id est laborum.
\section{Software requirements specification}

\end{document}